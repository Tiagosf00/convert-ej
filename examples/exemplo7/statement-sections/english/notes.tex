No primeiro caso de teste, existe um saco no baú que apresentam dois objetos. Como estes objetos estão no interior do saco, são considerados pedras preciosas, logo, existem duas pedras preciosas.

No segundo caso de teste, o baú apresenta dois sacos, sendo um saco externo que possui dois objetos e um outro saco interno, com outros dois objetos em seu interior. Por isso, podem ser contabilizadas 4 pedras preciosas. Existe um objeto externo que não é pedra preciosa, pois não está no interior de um saco.

No terceiro caso de teste, o baú apresenta apenas um saco, sem objetos em seu interior, e um pedaço de saco. Logo, não existem pedras preciosas no baú.
