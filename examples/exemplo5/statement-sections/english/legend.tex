Uma crise na Nova Zelândia no último inverno, causada por uma longa estiagem e baixo nível de água das represas, levou o governo a adotar um esquema de contingência para economia de energia. Sabe-se que o país é dividido em $N$ regiões, sendo que cada uma é identificada por um valor inteiro entre $1$ e $N$. A estratégia de contingência determina que haja interrupção da distribuição de energia às regiões por um período específico, mas apenas uma região de cada vez.

Para fazer isso, a energia da região $1$ (Auckland) é sempre a primeira a ser cortada por ser a sede do Governo (precisam dar o exemplo à população). A seguir, define-se um número inteiro $M$ que determinará qual será a próxima região afetada. A escolha é simples, seguindo a ordem crescente dos números, escolhe-se a $M$-ésima região seguinte. Obviamente, as regiões que já foram afetadas não podem ser consideradas novamente neste processo. Por exemplo, se $N=17$ e $M=5$, a energia será desligada de acordo com a seguinte ordem: $1, 6, 11, 16, 5, 12, 2, 9, 17, 10, 4, 15, 14, 3, 8, 13, 7$.

Como a sede da Companhia de Energia Elétrica da Nova Zelândia (100Z) está localizada na região de Wellington (identificada como $13$), esta deve ser, obrigatoriamente, desligada por último. Por isso, para um dado número $N$, o número $M$ precisa ser cuidadosamente escolhido de modo que a região $13$ seja a última região afetada.

Escreva um programa que faça a leitura do número de regiões $N$ e determine o menor número $M$, garantindo que a região de Wellington seja a última a ser desligada.
