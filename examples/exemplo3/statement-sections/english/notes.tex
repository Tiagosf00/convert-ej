No primeiro caso de teste, o limite de peso da rodovia é $7$ e como $F=1$, cada veículo que passar pelo posto será fiscalizado, começando do primeiro veículo.

Os veículos 1, 2, 3 e 4 estão dentro do limite de peso estabelecido, então eles são fiscalizados e liberados. O tempo total para isso = 5 + 5 + 5 + 5 = 20 minutos.

O veículo $5$, possui $9kg$ está acima do limite do peso da rodovia. Ele descarta 2kg, e volta pra fila com o peso igual a $7kg$. O tempo total gasto nesse caso é 10 minutos. Como só existe esse veículo na fila, ele é fiscalizado novamente, e como está com o peso dentro do limite, ele é liberado, totalizando $5$ minutos.

Logo, o tempo total foi 35 minutos.

