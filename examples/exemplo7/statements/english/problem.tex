\begin{problem}{Ouro alienígena}{standard input}{standard output}{1 second}{256 megabytes}

Tubertino é um sujeito aventureiro e um exímio caçador de tesouros em ambientes hostis. Por isso, ele vive realizando trilhas em lugares perigosos atrás de pedras preciosas para se tornar cada vez mais rico. Certo dia, Tubertino almoçou em um restaurante em Alto Paraíso de Goiás e ouviu de uma senhora que alienígenas deixaram uma grande quantidade de pedras preciosas em uma caverna, mas que nenhum aventureiro decidiu explorá-la devido ao risco de existirem alienígenas vigiando a localidade. 

Tubertino decidiu ir até essa caverna e após uma caminhada de 4 dias, encontrou um enorme baú com diversos objetos em seu interior. Em meio à escuridão da caverna, Tubertino pôde identificar sacos amarrados de diversos tamanhos e outros objetos espalhados que aparentemente não possuíam forma específica de qualquer pedra preciosa conhecida. 

Elabore um programa que, dada a configuração dos objetos do baú, determine a quantidade de pedras preciosas que Tubertino consegue identificar, isto é, a quantidade de objetos que estão no interior dos sacos. Repare que alguns sacos podem estar incluídos em outros sacos e que podem haver pedaços de sacos rasgados e espalhados pelo baú.

\InputFile
A entrada consiste de uma expressão $s$, em que $2 \leq \arrowvert s \arrowvert \leq 200$ contendo os \textbf{apenas} símbolos $(,)$ e $*$, em que as extremidades do saco são identificadas exatamente por $($ e por $)$, enquanto que o símbolo $*$ está relacionado com um objeto, que pode ser pedra preciosa ou não. Um objeto $*$ é considerado uma pedra preciosa somente se estiver dentro de um saco. Por exemplo, um saco vazio é representado por $()$, enquanto que um saco com pedras é definido por $(*)$, $(**)$ e assim sucessivamente.

\OutputFile
Imprima a quantidade de pedras preciosas que Tubertino encontrará no baú.

\Examples

\begin{example}
\exmpfile{example.01}{example.01.a}%
\exmpfile{example.02}{example.02.a}%
\exmpfile{example.03}{example.03.a}%
\end{example}

\Note
No primeiro caso de teste, existe um saco no baú que apresentam dois objetos. Como estes objetos estão no interior do saco, são considerados pedras preciosas, logo, existem duas pedras preciosas.

No segundo caso de teste, o baú apresenta dois sacos, sendo um saco externo que possui dois objetos e um outro saco interno, com outros dois objetos em seu interior. Por isso, podem ser contabilizadas 4 pedras preciosas. Existe um objeto externo que não é pedra preciosa, pois não está no interior de um saco.

No terceiro caso de teste, o baú apresenta apenas um saco, sem objetos em seu interior, e um pedaço de saco. Logo, não existem pedras preciosas no baú.

\end{problem}

