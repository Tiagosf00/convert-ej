Dona Rosângela está escrevendo um livro matemático contendo apenas em expressões matemáticas em notação pós-fixa. Sabemos que as expressões aritméticas que estudamos estão em notação infixa, isto é, em operações matemáticas binárias, o operador $y$ é apresentado entre dois operandos ($o_1$ e $o_2$), sob a forma $o_1 y o_2$.

Em seus estudos, Dona Rosângela considera as seguintes operações matemáticas binárias:

\begin{itemize}
\item \^ (potência);
\item / (divisão);
\item * (multiplicação);
\item - (subtração);
\item + (soma).
\end{itemize}

Além disso, os operandos podem ser representados por variáveis minúsculas (de $a$ até $z$) ou constantes de um único dígito numérico. Nessas expressões, podem existir parênteses para preservar a prioridade de determinadas operações, uma vez que alguns operadores possuem prioridade na ausência de parênteses (por exemplo, multiplicação tem prioridade em relação à soma e subtração).

Por exemplo, a expressão em notação pós-fixa ab*cd+2\^/ é obtida a partir da conversão da expressão em notação infixa a*4/(c+d)\^2. Ajude Dona Rosângela nessa tarefa e elabore um programa que converta uma expressão em notação infixa para a notação pós-fixa.
